\section{Some DFT Properties}\label{sec:p5}

\begin{enumerate}[(a)]
%----------------------------------------------------------------------------------------------
\item Define $k \bmod N$ as $\langle k \rangle_N$, i.e.
\[\langle k + N \rangle_N = \langle k \rangle_N.\]

The DFT of $x[n]$ is defined as
\[X[k] = \sum_{n=0}^{N-1} x[n] W_N^{-kn}\]
where
\[W_N = e^{j\frac{2\pi}{N}} = \cos(2\pi/N) + j\sin(2\pi /N).\]
Assume that $k = lN+r \Leftrightarrow \langle k \rangle_N = r$, then
\begin{align*}
	W_N^k
	&= \exp(j \frac{2\pi}{N} (lN+r)) \\
	&= \exp(j \frac{2\pi}{N} lN) \exp(j \frac{2\pi}{N} r) \\
	&= 1 \exp(j \frac{2\pi}{N} r) \\
	&= W_N^r = W_N^{\langle k \rangle_N}.
\end{align*}
Similarly, $W_N^{mk} = W_N^{m \langle k \rangle_N}$.

Furthermore, 
\begin{align*}
	X[k]
	&= \sum_{n=0}^{N-1}x[n] W_N^{-n(lN+r)} \\
	&= \sum_{n=0}^{N-1} x[n] W_N^{-nlN} W_N^{-nr} \\
	&= \sum_{n=0}^{N-1} x[n] W_N^{-nr} \\
	&= X[r] = X[\langle k \rangle_N].
\end{align*}

Therefore,
\begin{align*}
	DFT(x[\langle -n \rangle_N])
	&= \sum_{n=0}^{N-1}x[\langle -n \rangle_N] W_N^{-nk} \\
	&= \sum_{m=0}^{N-1} x[m] W_N^{-\langle -m\rangle_N k} \\
	&= \sum_{m=0}^{N-1} x[m] W_N^{mk} \\
	&= X[-k] \\
	&= X[\langle -k \rangle_N].
\end{align*}

%----------------------------------------------------------------------------------------------
\item The circular convolution between $x$ and $y$ can be defined as
\begin{align*}
	(x \circledast y)[n] 
	&= \sum_{m=0}^{N-1} x[m] y[\langle n - m\rangle_N].
\end{align*}
Therefore
\begin{align*}
	DFT((x \circledast y)[n])
	&= \sum_{n=0}^{N-1} \sum_{m=0}^{N-1} x[m] y[\langle n-m\rangle_N] W_N^{-nk} \\
	&= \sum_{m=0}^{N-1} x[m] \sum_{n=0}^{N-1} y[\langle n-m \rangle_N]W_N^{-nk} \\
	&= \sum_{m=0}^{N-1} x[m] W_N^{-mk} Y[k] \\
	&= Y[k] \sum_{m=0}^{N-1} x[m] W_N^{-nk} \\
	&= Y[k] X[k]
\end{align*}

%----------------------------------------------------------------------------------------------
\item The inverse DFT is defined as
\[x[n] = \frac{1}{N} \sum_{k=0}^{N-1} X[k] W_N^{kn}\]
Therefore
\begin{align*}
	IDFT((X \circledast Y)[n])
	&= \frac{1}{N} \sum_{k=0}^{N-1} (X \circledast Y)[k] W_N^{kn} \\
	&= \frac{1}{N} \sum_{k=0}^{N-1} \left( X[m] Y[\langle k - m \rangle_N]\right) W_N^{kn} \\
	&= \frac{1}{N} \sum_{k=0}^{N-1} \left( X[m] Y[\langle k - m \rangle_N]\right) W_N^{mn} W_N^{(k-m)n} \\
	&= \frac{1}{N} \sum_{m=0}^{N-1} X[m]W_N^{mn} \sum_{k=0}^{N-1}Y[\langle k-m \rangle_N] W_N^{(k-m)n} \\
	&= x[n]y[n].
\end{align*}

%----------------------------------------------------------------------------------------------
\item If $x[n]$ is real,
\begin{align*}
	&\Rightarrow x[n] = x^*[n] \\
	&\Rightarrow DFT(x[n]) = DFT(x^*[n]) \\
	&\Rightarrow X[k] = X^*[\langle -k \rangle_N].
\end{align*}

If $x[n]$ is also symmetric,
\begin{align*}
	&\Rightarrow x[n] = x[\langle -n \rangle_N] \\
	&\Rightarrow DFT(x[n]) = DFT(x[\langle -n \rangle_N]) \\
	&\Rightarrow X[k] = X[\langle -k \rangle_N].
\end{align*}

Therefore
\[X[k] = X^*[\langle -k \rangle_N] = X[\langle -k \rangle_N].\]
Hence, $X[k]$ is real (and also symmetric.)

%----------------------------------------------------------------------------------------------
\item If $x[n]$ is symmetric (and real),
\begin{align*}
&\Rightarrow x[n] = -x[\langle -n \rangle_N] \\
&\Rightarrow DFT(x[n]) = -DFT(x[\langle -n \rangle_N]) \\
&\Rightarrow X[k] = -X[\langle -k \rangle_N].
\end{align*}

Therefore
\[X[k] = X^*[\langle -k \rangle_N] = -X[\langle -k \rangle_N].\]
Hence, $X[k]$ is imaginary.
\end{enumerate}