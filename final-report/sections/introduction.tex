\section{Introduction and background}\label{sec:intro}
Object recognition is a problem of computer vision, where the objective is to determine objects in an image or video sequence. This task has multiple applications in real life, such as in autonomous vehicles, where pedestrians, stop signs, traffic lights, etc. are automatically recognized, helping the cars react accordingly. It can be used in security systems, such as video surveillance or face recognition.

There are different researches that try to solve this problem. The classical approach is by using hand crafted features. Such features are extracted from images and clustered together such that each cluster represent a specific aspect of the objects. The process's objective is to create a visual dictionary to represent training objects. New objects are classified based on their features, which are extracted in the same manner as training samples, using nearest neighbors approach \cite{lazebnik_bof}. With the advance of technologies and algorithms, solutions for object recognition are gradually shifting to deep learning methods, where extracted features are machined-based instead of hand-crafted-based, i.e. features are understandable from computer's perspective instead of human. There are several network architectures, started by AlexNet that won the ImageNet challenge and set the new face for object recognition. Some networks recently such as VGG Net \cite{Simonyan2014_vgg}, GoogleNet \cite{googlenet}, and ResNet \cite{resnet} are boosting the architecture complexity and achieving state-of-the-art performance.

Color images suffer from numerous problems, such as inconsistent lighting condition, noisy background data, etc. that reduce system's robustness. Such problems can be solved by using depth information, since depth images contain more information about the shape of objects and ignore patterns on the surface. Therefore, depth images have higher consistency among objects from the same category. In addition, depth cameras are getting more popular therefore such images are easily obtainable.

In this project, I propose to reproduce Eitel et al. paper ``Multimodal deep learning for robust RGB-D object recognition'' \cite{Eitel2015} to build an object recognition system that combines color and depth information to improve accuracy, implemented using Google's TensorFlow \cite{tensorflow2015-whitepaper}. Depth information in converted into RGB images and fed in a branch of the network along side with normal color information. They are then fused and recomputed to produced the final results.