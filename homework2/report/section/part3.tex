\section{Orthogonalization of a Projection}\label{sec:part3}

\begin{enumerate}[(a)]
\item If $P = P^*$,
\[\innerprod{Px}{y}_0 = y^* (Px) = (y^*P)x = (y^*P^*)x = (Py)^*x = \innerprod{x}{Py}_0, \qquad \forall x, y \in V.\]
Hence, $P$ is self-adjoint with respect to $\innerprod{\cdot}{\cdot}$ on $\mathbb{C}^N$ if $P = P^*$.

\item Remind that a non-zero vector $v \in \mathbb{C}^N$ is an eigenvector of square matrix $P \in \mathbb{C}^{N\times N}$ if $Pv = \lambda v$, where $\lambda$ is the eigenvalue associated with $v$. 

If $P = P^2$,
\[\lambda v = P v = P^2 v = \lambda^2 v\]
Since $v \neq 0$, $\lambda = \lambda^2 \Leftrightarrow \lambda = 0 \text{ or } \lambda = 1$. Hence the eigenvalues of an oblique projection is 0 or 1.

\item Since $P = T^{-1}DT$, where $D$ is a diagonal matrix with eigenvalues found in part (b), $D$'s diagonal is formed by 1 and 0. Therefore, $D^* = D$

Since $\innerprod{x}{y}_T \triangleq y^*T^*Tx$, we have
\begin{align*}
	\innerprod{Px}{y}_T
	&= y^* T^* T P x \\
	&= y^* T^* T (T^{-1} D T) x \\
	&= y^* T^* D T x \\
	&= y^* T^* D I^* T x \\
	&= y^* T^* D (T T^{-1})^* T x \\
	&= y^* T^* D (T^{-1})^* T^* T x \\
	&= y^* T^* D^* (T^{-1})^* T^* T x \\
	&= y^* (T^{-1}DT)^* T^* T x \\
	&= y^* P^* T^* T x \\
	&= (Py)^* T^* T x \\
	&= \innerprod{x}{Py}_T
\end{align*}

\item We have $P = T^{-1}DT \Rightarrow P^* = T^* D^* (T^{-1})^*$.
\begin{align*}
	\innerprod{x-Px}{Px}_T 
	&= x^* P^* T^* T (x - Px) = x^* P^* T^* T x - x^* P^* T^* T P x \\
	&= x^* T^* D^* (T^{-1})^* T^* T x - x^* T^* D^* (T^{-1})^* T^* T T^{-1}DT x \\
	&= x^* T^* D^* ((T^{-1})^* T^*) T x - x^* T^* D^* ((T^{-1})^* T^*) (T T^{-1})DT x \\
	&= x^* T^* D^* T x - x^* T^* D^* DT x
\end{align*}
Since $D$ is a diagonal matrix with only 1 and 0, $D D^* = D^* D = D = D^*$. Therefore,
\begin{align*}
	\innerprod{x-Px}{Px}_T 
	&= x^* T^* D^* T x - x^* T^* D^* DT x \\
	&= x^* T^* D^* T x - x^* T^* D^* T x = 0
\end{align*}
Hence, $x-Px \perp Px$.

\item From part (c), we know that $P$ is oblique and self-adjoint, i.e. $P = P^2 = P^*$. Since $P \in \mathbb{C}^{N \times N}$, $I = I_N$ and $PI = IP = P$. Therefore,
\[(I-P)^2 = (I-P)(I-P) = I^2 - IP - PI + P^2 = I - 2P + P^2 = I - 2P + P = I - P\]
and
\[(I-P)^* = I^* - P^* = I - P\]
Hence, $I-P$ is oblique and self-adjoint.

We know that for any matrix $A$, $\mathcal{R}(A) \perp \mathcal{N}(A^\top)$. Therefore,
\[\mathcal{R}(I-P) \perp \mathcal{N}((I-P)^\top)\]
Since $I-P$ is proven to be self-adjoint, $(I-P)^\top = I-P$
\[\Rightarrow \mathcal{R}(I-P) \perp \mathcal{N}(I-P)\]

From the definition of null space:
\begin{align*}
	\mathcal{N}(P) &= \{x \mid Px = 0\} \\
	\mathcal{N}(I-P) &= \{x \mid (I-P)x = 0\}
\end{align*}
We proved that $x-Px \perp Px$ wrt $\innerprod{\cdot}{\cdot}_T$, so 
\[(I-P)x \perp Px \Rightarrow \mathcal{N}(I-P) \perp \mathcal{N}(P)\]
We already have $\mathcal{R}(I-P) \perp \mathcal{N}(I-P)$, thus $\mathcal{R}(I-P) = \mathcal{N}(P)$.

\end{enumerate}