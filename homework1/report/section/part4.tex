\section{Signal Sets and Spaces}\label{sec:part4}

\subsection{}\label{subsec:p4-a}
For $S = \mathbb{R}$, we have
\[\mathbb{R}^I = \left\{v \mid v: I \rightarrow \mathbb{R} \right\}.\]
We know that $\mathbb{R}$ is closed under addition and scalar multiplication ($\mathbb{R}$ is a vector space.) Therefore, for $t \in I$ and $u,v \in \mathbb{R}^I$,
\begin{align*}
	&u[t], v[t] \in \mathbb{R} \\
	\Rightarrow &u[t] + v[t] \in \mathbb{R} \\
	\Rightarrow &(u+v)[t] := u[t] + v[t] \in \mathbb{R} .
\end{align*}
For scalar $\alpha \in \mathbb{R}$,
\begin{align*}
	&\alpha u[t] \in \mathbb{R} \\
	\Rightarrow &(\alpha u)[t] := \alpha u[t] \in \mathbb{R}
\end{align*}


\subsection{}\label{subsec:p4-b}
\begin{enumerate}[(i)]
	\item For complex-valued sequences indexed by the integers, signal values live in complex space and indices live in integer space, i.e.
	\[\mathbb{C}^I = \left\{ v \mid v: \mathbb{Z} \rightarrow \mathbb{C}\right\}. \]
	Since $\mathbb{C}$ is also a vector space, this signal set is linear. The proof is similar as in \ref{subsec:p4-a}.
	
	Zero vector in $\mathbb{R}^I$ is defined as $\{ a_i \}_{i \in I}$, where $a_i = 0, \forall i \in I$.
	
	\item For 8-bit RGB color (three channels) digital photos of dimension $W \times H$, we can choose the value set to be $\mathcal{B}_8 = \{\overline{x_0 x_1 ... x_7}\}, x_i = \{0,1\}, i = \{0,7\}$, denoting the set of all binary sequences with length of 8. The indices are chosen as set of 3 natural numbers, corresponding to the width, height, and channel, i.e.
	\[\mathcal{B}_8^I = \left\{ v \mid v : \mathbb{N}^{W \times H \times 3} \rightarrow \mathcal{B}_8\right\}.\]

	This set is not linear because it is not closed under addition, e.g. $(11111111)_2 + (00000001)_2 = (100000000)_2 \notin \mathcal{B}_8$. We can also see that , in general, $\mathcal{B}_k$ with finite $k$ is not a linear space.
	
	\item For 32-bit floating point buffers containing 1 second of stereo audio at 48KHz, we can choose the value space as $\mathcal{B}_{32}$, with similar definition as $\mathcal{B}_8$. Since the audio is 1-second long with 48KHz, there are 48k samples. Therefore, the signal set can be described as:
	\[\mathcal{B}_{32}^I = \left\{v \mid v : \mathbb{N}^{48K \times 2} \rightarrow \mathcal{B}_{32}\right\}.\]
	
	We can see that $\mathcal{B}_{32}$ is not a linear, therefore the signal set is not linear.
\end{enumerate}


\subsection{}\label{subsec:p4-c}
We can consider signals with indices $I_1 \subset I_2$ are truncated version of signals with indices $I_2$, where the signals in $I_d = I_2 / I_1$ are reduced to 0. So $\mathbb{R}^{I_1}$ is a subset of $\mathbb{R}^{I_2}$. Since $\mathbb{R}$ is a subspace, $\mathbb{R}^{I_1}$ is closed under addition and scalar multiplication. Hence, $\mathbb{R}^{I_1}$ is a subspace of $\mathbb{R}^{I_2}$.


\subsection{}\label{subsec:p4-d}
\paragraph{Linearity} Let $u, v \in \mathbb{R}^I$ and $\alpha, \beta$ be scalars
\begin{align*}
	T(\alpha u + \beta v)_k = T(\alpha u)_k + T(\beta v)_k = \alpha (Tu)_k + \beta (T_v)_k.
\end{align*}
Hence, $T$ is linear.

\paragraph{Invertibility} Let $Tu = v$. $T$ can be seen as a permutation matrix that maps the $k$ entry of $v$ to $i_k$ entry of $u$. Therefore $T$ has full rank and $\rank(T) = N$ (there are $N$ indices). Hence, $T$ is invertible (full rank matrices are invertible.)


\subsection{}\label{subsec:p4-e}
We see that $u[i], v[i] \in \mathbb{R}, \forall i \in I$. We need to prove that the defined inner product satisfies the three axioms:

\paragraph{Conjugate symmetry:}
\[\overline{\innerprod{v}{u}}_I = \overline{\sum_{i \in I} u[i]v[i]} = \sum_{i \in I} \overline{u[i]v[i]} = \sum_{i \in I} u[i]v[i] = \innerprod{u}{v}_I.\]

\paragraph{Linearity in the first argument:}
\[\innerprod{\alpha u}{v} = \sum_{i \in I} \alpha u[i]v[i] = \alpha \sum_{i \in I} u[i]v[i] = \alpha \innerprod{u}{v}.\]
Let $u_1, u_2 \in \mathbb{R}^I$,
\[\innerprod{u_1+u_2}{v} = \sum_{i \in I} (u_1[i]+u_2[i])v[i] = \sum_{i \in I} u_1[i]v[i] + \sum_{i \in I} u_2[i]v[i] = \innerprod{u_1}{v} + \innerprod{u_2}{v}.\]

\paragraph{Positive-definiteness:}
\[\innerprod{u}{u} = \sum_{i \in I} u[i]u[i] \sum_{i \in I} (u[i])^2 \geq 0\]
Let $u$ be a zero vector, i.e. $u[i] = 0, \forall i \in I$,
\[\innerprod{u}{u} = \sum_{i \in I} (u[i])^2 = \sum_{i \in I} 0 = 0.\]

Hence, $\innerprod{u}{v}_I := \sum_{i \in I} v[i]u[i]$ is an inner-product in $\mathbb{R}^I$.

\subsection{}\label{subsec:p4-f}
For an arbitrary $t \in I$, let $e_t = \{\epsilon_\tau\}_{\tau \in I}$, s.t.
\[
\epsilon_\tau = \begin{cases}
1, &\tau = t\\
0, &\text{otherwise}.
\end{cases}
\]
It is obvious that $\{e_\tau\}$ are linearly independent, as $\sum_{\tau} \alpha_\tau e_\tau = 0 \Leftrightarrow \alpha_\tau =0, \forall \tau$. Therefore $\{e_\tau\}$ is a basis. Since $e_t \subset \mathbb{R}^I$, we have
\[\innerprod{u}{e_t}_I = \sum_{\tau \ in I} u[\tau] \epsilon_\tau = u[t].\]
which satisfies the criteria. Hence, the defined $e_t$ is the standard basis (or reproducing kernel).