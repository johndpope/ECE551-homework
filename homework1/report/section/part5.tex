\section{Computer Exercise}\label{sec:part5}
This section include the implementation using Python 2.7. The real code can be found in \texttt{main.py}.

\subsection{}\label{subsec:p5-a}
\lstinputlisting[language=Python, style=codestyle, firstline=9, lastline=20]{main.py}
\lstinputlisting[language=Python, style=codestyle, firstline=113, lastline=114]{main.py}

\subsection{}\label{subsec:p5-b}
\lstinputlisting[language=Python, style=codestyle, firstline=117, lastline=121]{main.py}

\subsection{}\label{subsec:p5-c}
\lstinputlisting[language=Python, style=codestyle, firstline=23, lastline=49]{main.py}
\lstinputlisting[language=Python, style=codestyle, firstline=124, lastline=126]{main.py}

\subsection{}\label{subsec:p5-d}
\lstinputlisting[language=Python, style=codestyle, firstline=52, lastline=65]{main.py}
\lstinputlisting[language=Python, style=codestyle, firstline=129, lastline=132]{main.py}

\paragraph{Bonus} Assume that we do not change the order of index set, the mapping $T: \mathbb{R}^I \rightarrow \mathbb{R}^N$ is implemented as:
\lstinputlisting[language=Python, style=codestyle, firstline=68, lastline=92]{main.py}

\subsection{}\label{subsec:p5-e}
\lstinputlisting[language=Python, style=codestyle, firstline=95, lastline=108]{main.py}
\lstinputlisting[language=Python, style=codestyle, firstline=141, lastline=146]{main.py}